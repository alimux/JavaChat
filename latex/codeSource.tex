\documentclass[a4paper,12pt]{report}
\usepackage{amssymb} % needed for math
\usepackage{amsmath} % needed for math
\usepackage[utf8]{inputenc} % this is needed for fr umlauts
\usepackage[french]{babel} % this is needed for fr umlauts
\usepackage[T1]{fontenc}    % this is needed for correct output of umlauts in pdf
\usepackage{textcomp}
\usepackage[margin=2.5cm]{geometry} %layout
\usepackage{listings} % needed for the inclusion of source code
\usepackage[usenames,dvipsnames,svgnames,table]{xcolor}
\usepackage[pdftex,urlcolor=blue,pdfstartview=FitH]{hyperref}


\lstset{ %
  language=Java,                  % the language of the code
  basicstyle=\footnotesize,       % the size of the fonts that are used for the code
  numbers=left,                   % where to put the line-numbers
  numberstyle=\tiny\color{gray},  % the style that is used for the line-numbers
  stepnumber=1,                   % the step between two line-numbers. If it's 1, each line 
                                  % will be numbered
  numbersep=5pt,                  % how far the line-numbers are from the code
  backgroundcolor=\color{white},  % choose the background color. You must add \usepackage{color}
  showspaces=false,               % show spaces adding particular underscores
  showstringspaces=false,         % underline spaces within strings
  showtabs=false,                 % show tabs within strings adding particular underscores
  frame=single,                   % adds a frame around the code
  rulecolor=\color{black},        % if not set, the frame-color may be changed on line-breaks within not-black text (e.g. commens (green here))
  tabsize=4,                      % sets default tabsize to 2 spaces
  % captionpos=b,                   % sets the caption-position to bottom
  breaklines=true,                % sets automatic line breaking
  breakatwhitespace=false,        % sets if automatic breaks should only happen at whitespace
  title=\lstname,                 % show the filename of files included with \lstinputlisting;
                                  % also try caption instead of title
  keywordstyle=\color{blue},          % keyword style
  commentstyle=\color{olive},       % comment style
  stringstyle=\color{violet},         % string literal style
  escapeinside={\%*}{*)},            % if you want to add a comment within your code
  morekeywords={*,...}               % if you want to add more keywords to the set
  inputencoding=utf8,
  extendedchars=true, %To use uft8
  literate={á}{{\'a}}1 {ã}{{\~a}}1 {é}{{\'e}}1,
}

% this is needed for forms and links within the text
\usepackage{hyperref}  

%%%%%%%%%%%%%%%%%%%%%%%%%%%%%%%%%%%%%%%%%%%%%%%%%%%%%%%%%%%%%%%%%%%%%%
% Variablen                                                          %
%%%%%%%%%%%%%%%%%%%%%%%%%%%%%%%%%%%%%%%%%%%%%%%%%%%%%%%%%%%%%%%%%%%%%%
\newcommand{\authorName}{Pierre Labadille, Alexandre Ducreux}
\newcommand{\tags}{\authorName, java, chat, M2-DNR2i}
\title{Code source : Java Chat}
\title{%
  Projet Java: Java Chat \\
  \large Document annexe: Code Source \\
    Sujet proposé par Yann Mathet \\
    Université de Caen Normandie - M2-DNR2i}
\author{\authorName}
\date{\today}

%%%%%%%%%%%%%%%%%%%%%%%%%%%%%%%%%%%%%%%%%%%%%%%%%%%%%%%%%%%%%%%%%%%%%%
% PDF Meta information                                               %
%%%%%%%%%%%%%%%%%%%%%%%%%%%%%%%%%%%%%%%%%%%%%%%%%%%%%%%%%%%%%%%%%%%%%%
\hypersetup{
  pdfauthor   = {\authorName},
  pdfkeywords = {\tags},
  pdftitle    = {sourceCodeJavaChat_cc2017}
} 

%%%%%%%%%%%%%%%%%%%%%%%%%%%%%%%%%%%%%%%%%%%%%%%%%%%%%%%%%%%%%%%%%%%%%%
% THE DOCUMENT BEGINS                                                %
%%%%%%%%%%%%%%%%%%%%%%%%%%%%%%%%%%%%%%%%%%%%%%%%%%%%%%%%%%%%%%%%%%%%%%
\begin{document}
  \maketitle
  \tableofcontents

  \chapter{Client}
    \section{Package src/dnr2i/chat}
      \subsection{Package gui}
        \lstinputlisting[language=Java]{../client/src/dnr2i/chat/gui/JavaChat.java}
        \lstinputlisting[language=Java]{../client/src/dnr2i/chat/gui/GUIJavaChat.java}
        \lstinputlisting[language=Java]{../client/src/dnr2i/chat/gui/TopPanel.java}
        \lstinputlisting[language=Java]{../client/src/dnr2i/chat/gui/DownPanel.java}
        \lstinputlisting[language=Java]{../client/src/dnr2i/chat/gui/UsersPanel.java}
        \lstinputlisting[language=Java]{../client/src/dnr2i/chat/gui/GUICircumscribedArea.java}
        \lstinputlisting[language=Java]{../client/src/dnr2i/chat/gui/Constants.java}
        \lstinputlisting[language=Java]{../client/src/dnr2i/chat/gui/socket/Connection.java}
      \subsection{Package manager}
        \lstinputlisting[language=Java]{../client/src/dnr2i/chat/manager/ChatManager.java}
        \lstinputlisting[language=Java]{../client/src/dnr2i/chat/manager/ClientIncomingDirective.java}
        \lstinputlisting[language=Java]{../client/src/dnr2i/chat/manager/Message.java}
      \subsection{package user}
        \lstinputlisting[language=Java]{../client/src/dnr2i/chat/user/User.java}
    \section{Package src/dnr2i/util}
      \subsection{Package event}
        \lstinputlisting[language=Java]{../client/src/dnr2i/util/event/IListenableModel.java}
        \lstinputlisting[language=Java]{../client/src/dnr2i/util/event/ListenableModel.java}
        \lstinputlisting[language=Java]{../client/src/dnr2i/util/event/ListenerModel.java}

  \chapter{Server}
    \section{Package dnr2i/chatServer}
      \lstinputlisting[language=Java]{../server/dnr2i/chatServer/StartServer.java}
      \lstinputlisting[language=Java]{../server/dnr2i/chatServer/ChatServer.java}
      \lstinputlisting[language=Java]{../server/dnr2i/chatServer/ChatServerConnection.java}
      \lstinputlisting[language=Java]{../server/dnr2i/chatServer/ChatServerDirectiveManager.java}

\end{document}